% Created 2014-10-30 Thu 10:43
\documentclass[a4paper,times,12pt,listings-bw,microtype]{article}
\usepackage[utf8]{inputenc}
\usepackage[T1]{fontenc}
\usepackage{fixltx2e}
\usepackage{graphicx}
\usepackage{longtable}
\usepackage{float}
\usepackage{wrapfig}
\usepackage{rotating}
\usepackage[normalem]{ulem}
\usepackage{amsmath}
\usepackage{textcomp}
\usepackage{marvosym}
\usepackage{wasysym}
\usepackage{amssymb}
\usepackage{hyperref}
\tolerance=1000
\usepackage{longtable,tabulary,booktabs,threeparttable,tabularx,graphicx,svg,float,wrapfig,url,underscore}
\usepackage{parnotes,amsmath,amssymb,marvosym,wasysym}
\usepackage[citestyle=authoryear-icomp,bibstyle=authoryear,hyperref=true,maxcitenames=3,url=true,backend=biber,natbib=true]{biblatex}
\usepackage[section,below]{placeins}
\usepackage[dvipsnames,svgnames,table]{xcolor}
\usepackage[innermargin=1.5in,outermargin=1.25in,vmargin=1.25in]{geometry}
\usepackage[nomain,acronym,xindy,toc]{glossaries}
\hypersetup{colorlinks=true,citecolor=blue,linkcolor=blue,citebordercolor={0 1 0},linktocpage,pdfstartview=FitH,anchorcolor=blue,filecolor=blue,menucolor=blue,urlcolor=blue}
\linespread{1.3}
\setcounter{secnumdepth}{3}
\author{Bingwei Tian  \thanks{bwtian@gmail.com}\\  \small{Kyoto University, Kyoto, Japan}}
\date{}
\title{glossaR}
\hypersetup{
  pdfkeywords={},
  pdfsubject={},
  pdfcreator={Emacs 24.3.1 (Org mode 8.2.10)}}
\begin{document}

\maketitle
\setcounter{tocdepth}{2}
\tableofcontents

\label{sec-1}
\begin{abstract}
Abstract:
\end{abstract}


\section{fig:glossarworkflow}
\label{sec-2}
\section{fig:work with R to extract glossaries}
\label{sec-3}
\section{R:}
\label{sec-4}

\section{tbl:Restec}
\label{sec-5}
\begin{verbatim}
###############################################################################
## R code chunk:
###############################################################################
\end{verbatim}

\section{Remote Sensing Glossaries}
\label{sec-6}

\subsection{\url{http://www.ldeo.columbia.edu/res/fac/rsvlab/glossary.html}}
\label{sec-6-1}
\begin{verbatim}
###############################################################################
## R code chunk:
###############################################################################
\end{verbatim}
Emacs 24.3.1 (Org mode 8.2.10)
\end{document}